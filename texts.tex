\section{\esp Introdução}
Com a mudança de mentalidade da \textit{Web} para o novo paradigma de aplicações conhecido como \textit{Web} 2.0 proposto por 
\citeonline{web20Proposta}, os 
usuários deixaram de ter um papel passivo como consumidores de conteúdo para assumir um papel ativo de produtores de conteúdo. Esta mudança 
trouxe consigo ferramentas interativas, que têm como foco principal a geração de conteúdo por parte dos próprios usuários, tais como fóruns 
de discussões e, recentemente as redes sociais como o Twitter, Facebook e o Instagram.

Esta participação ativa dos usuários impactou significativamente a quantidade de informação disponível na \textit{Web}. Segundo \citeonline{artigo01}, 
o tráfego de dados nesta rede dobra a cada ano, o que gera uma quantidade massiva de informação, tornando a \textit{Web} muito influente em diversos 
setores de grande importância na sociedade, passando a fazer parte do cotidiano das pessoas que estão cada vez mais conectadas a mecanismos que
as  conectam aos conteúdos de seu interesse. Pode-se perceber esta influência principalmente por meio das redes sociais, que segundo 
\citeonline{redesSociais01} estão mudando profundamente o modo de organização, identidade, conversação e mobilização social.

Como descreve \citeonline{deitelAjax}, com todo este conteúdo disponível, os mecanismos de busca de informação 
acabam recebendo destaque nesta ``nova \textit{Web}'' e sobressaem aqueles que auxiliam os usuários  a localizar e filtrar 
as informações desejadas. Deste modo, há um grande desafio em relação à formulação de ferramentas que guie o usuário até os conteúdos de seu 
interesse, economizando assim sua atenção. Estas ferramentas, com o passar do tempo, se fazem mais presentes no dia-a-dia das pessoas 
que, por sua vez, demandam por mecanismos cada vez mais robustos e personalizados de acordo com suas necessidades. Como consequência, existe 
uma demanda constante por aplicações mais específicas para atender determinada comunidade.

No estudo estudo realizado por \citeonline{perfilCicloturista}, após realizar uma pesquisa com 302 cicloturistas, concluiu-se que a \textit{internet}
é, sem dúvidas, 
o meio de informação mais utilizado por esta comunidade, totalizando 46\% das respostas em relação a qual o meio de informação utilizado por eles.
No entanto, neste meio, há uma demanda ainda não explorada por uma aplicação focada nesta prática, pois de acordo com \citeonline{cicloturismo02}, 
as grandes viagens realizadas sobre a bicicleta requerem melhor preparo e conhecimento por parte do ciclista, e ressalta a importância de  
conhecer a extensão da viagem e o tempo total disponível, além da região, relevo e clima escolhidos como trajeto. 
Devido a complexidade do planejamento
requerido, percebe-se, neste meio, a necessidade de uma ferramenta que centralize as informações importantes para tal. 
Praticantes desta atividade, por não 
possuírem uma ferramenta específica, recorrem a outras com propósitos similares para auxiliá-los no planejamento de seus trajetos, que não oferecem 
informações importantes para o mesmo. O cicloturismo, conforme descrito por \citeonline{cicloturismo01}, é uma modalidade do ecoturismo que está 
ganhando cada vez mais adeptos no Brasil, por ser uma atividade de baixo impacto ambiental, já que é realizada com bicicletas.

Este trabalho tem como objetivo desenvolver uma ferramenta no formato rede social \textit{Web} para atender esta demanda de 
informações observada 
no meio do cicloturismo. Segundo \citeonline{perfilCicloturista}, o  "fenômeno" cicloturismo é considerado de fraco a bom em termos de comunicação 
e estrutura para receber os cicloturistas. Com o Ciclotour, visamos prover um meio de comunicação onde os
usuários terão acesso de 
forma centralizada a informações sobre rotas que pretendem trilhar, fornecidas por outros usuários de modo colaborativo, com o propósito 
de criar um grande repositório de informações relevantes para os cicloturistas, auxiliando no planejamento da atividade turística e comunicação 
entre praticantes e promoção da prática.

\section{\esp Referencial Teórico}
\subsection{Arquitetura \textit{Web} e Ajax}
No princípio das aplicações \textit{Web}, devido ao seu comportamento síncrono, as mesmas ficaram para trás em relação às 
aplicações \textit{desktop}, pois a 
interação com essas aplicações resultava em um longo período de espera conforme descreve \citeonline{deitelAjax}. Este atraso era causado devido 
à grande 
quantidade de atualizações de páginas inteiras necessárias no ciclo síncrono, demonstrado na Figura \ref{fig:arquitetura_web_tradicional}. 

\begin{figure}[!ht]
	\centering	
	\caption[\hspace{0.1cm}Aplicação \textit{Web} clássica.]{Aplicação \textit{Web} clássica}
	  \vspace{-0.4cm}
	\includegraphics[width=.8\textwidth]{figuras/arquitetura_web_tradicional.png}
	 \vspace{-0.3cm}
	\\\textbf{\footnotesize Fonte: \citeauthor{deitelAjax}, \citeyear{deitelAjax}}
	\label{fig:arquitetura_web_tradicional}
\end{figure}

Cada recarga na página gerava uma nova requisição ao servidor que, neste modelo, é responsável por processar a requisição, gerar e enviar a resposta 
contendo a página exata que será exibida no \textit{browser} do cliente. Esse \textit{delay}, presente nas interações com a aplicação, 
fizeram os usuários ``exigirem'' um modo mais adequado de interação com estas aplicações. 

Para solucionar este problema de desempenho presente nas aplicações tradicionais, surgiu o modelo Ajax. Como descrito por 
\citeonline{garrettAjax}, uma 
aplicação Ajax elimina a natureza da interação \textit{Web} conhecida como \textit{start-stop-start-stop} introduzindo uma ,camada intermediária 
— \textit{Ajax Engine} — entre cliente e servidor. Essa camada permitiu a essas aplicações realizarem requisições ao servidor de forma assíncrona 
e atualizar parcialmente a página ao receber a resposta, não impedindo a interação do usuário durante o ciclo de requisição/resposta. Na 
Figura \ref{fig:ajax_comparison}, pode-se ver um comparativo entre aplicações \textit{Web} síncronas e assíncronas.

\begin{figure}[!ht]
	\centering	
	\caption[\hspace{0.1cm}Comparação aplicação \textit{Web} clássica e aplicação \textit{Web} com Ajax.]{Comparação aplicação \textit{Web} clássica e aplicação \textit{Web} com Ajax}
	  \vspace{-0.4cm}
	\includegraphics[width=.6\textwidth]{figuras/ajax_comparison.png}
	 \vspace{-0.3cm}
	\\\textbf{\footnotesize Fonte: \citeauthor{garrettAjax}, \citeyear{garrettAjax}}
	\label{fig:ajax_comparison}
\end{figure}


Com aplicações construídas utilizando Ajax, a \textit{Web} voltou a ser atrativa para o usuário e tem crescido o número de 
aplicações \textit{Web} desde então. Segundo
\citeonline{tabulaRest} as aplicações \textit{Web} assumem, atualmente, uma importância sem precedentes em todas as áreas da sociedade. 

\subsection{RESTful e \textit{Single Page Applications}}
Em \citeonline{fieldingRest} foi apresentado o \textit{Representational State Transfer} (REST) como um estilo arquitetural 
para sistemas de hipermídia 
distribuídos baseado no \textit{HyperText Transfer Protocol} (HTTP). Segundo \citeonline{modelingRestful}, o entendimento do REST 
pode ajudar na obtenção 
de uma melhor performance e escalabilidade, bem como diminuir o acoplamento, como resultado, aumentando a interoperabilidade, promovendo o reuso. 
Aplicações que seguem as restrições REST são denominadas como RESTful. Estas aplicações consistem 
em um \textit{\textit{Web} Service} simples que utiliza 
verbos HTTP para mapear ações \textit{Create, Read, Update, Delete} (CRUD). Pode-se ver, na Tabela \ref{tab:tabela1}, o mapeamento de verbos HTTP 
com as ações CRUD.

\begin{table}[htb]
	\centering
	\caption{\hspace{0.1cm} Relacionamento de verbos HTTP com ações CRUD}
	\vspace{-0.3cm} % espaço entre titulo e tabela
	\label{tab:tabela1}
	% Conteúdo da tabela
	\begin{tabular}{l|c}
  \hline
    \textbf{HTTP} & \textbf{Ação} \\
    \hline
      GET & Read \\
      POST & Create \\
      PUT & Update \\
      DELETE & Delete \\
     \hline
 \end{tabular}
 	\vspace{.1cm}  %espaço entre tabela e fonte
	\small
	% Fonte
	{\footnotesize\\ \textbf{Fonte: Elaborada pelo autor}}
\end{table}

Devido a este aprimoramento constante no que diz respeito ao desenvolvimento de aplicações \textit{Web}, a arquitetura de uma aplicação 
\textit{Web} está deixando 
de ser composta por grandes quantidades de códigos \textit{server-side}, onde todo o processamento é feito no servidor, os códigos são fechados e 
o consumo de banda é maior aos proprietários do sistema, e passando a ter maior utilização de códigos \textit{client-side}, onde o processamento dos 
\textit{scripts} da página ocorre na máquina do cliente e somente quando há necessidade de interação com o banco de dados são 
realizadas requisições ao 
servidor \cite{spa01}. Neste contexto, surgiram as \textit{Single Page Applications} (SPA) definidas por 
\citeonline{spa02}, aplicações compostas por somente 
uma página HTML, como base para todas as outras páginas \textit{Web} da aplicação, nas quais 
as interações feitas pelo usuário são implementadas usando 
JavaScript, HTML e CSS. A Figura \ref{fig:spa} apresenta o funcionamento de uma SPA.

\begin{figure}[!ht]
	\centering	
	\caption[\hspace{0.1cm} Ciclo de vida de uma \textit{Single Page Application}.]{Ciclo de vida de uma \textit{Single Page Application}}
	  \vspace{-0.4cm}
	\includegraphics[width=.6\textwidth]{figuras/spa.png}
	 \vspace{-0.3cm}
	\\\textbf{\footnotesize Fonte: \citeauthor{spa02}, \citeyear{spa02}}
	\label{fig:spa}
\end{figure}

Em aplicações SPA, a primeira requisição realizada pelo cliente recebe como resposta uma página HTML. 
Nesta página inicial há um extenso conteúdo de
\textit{scripts} responsáveis pela navegação do usuário, sem que haja recarga da página, quando existe a 
necessidade de conteúdos presentes no servidor. 
Esta comunicação é realizada por meio de requisições Ajax, e os scripts presentes na página \textit{Web}
são responsáveis por renderizar estes conteúdos recebidos na
página, por meio do \textit{Document Object Model} (DOM). Toda transição de páginas neste tipo de aplicação é 
realizada \textit{client-side}. 
Uma das vantagens da adoção deste modelo em relação ao modelo anterior é em relação ao \textit{overhead} de requisições, o que torna a aplicação 
mais leve e permite melhor usabilidade para o usuário final, devido ao Ajax não bloquear a manipulação da página durante a sua 
execução \cite{spa01}, 
além de facilitar a portabilidade da aplicação pois este modelo requer uma API \textit{server-side} robusta e bem estruturada para funcionar.

\section{\esp Trabalhos Relacionados}
\subsection{Runtastic}
A rede Runtastic \cite{runtastic}, reúne várias modalidades de esportes como ciclismo, corrida ou triatletismo. Oferece a seus usuários a
possibilidade de registrar suas atividades e performance obtida, com propósito de montar estatísticas que podem ser compartilhadas com 
os seus amigos e
usadas para medir o seu progresso. Esta rede social está associada a várias práticas físicas, com seu foco voltado ao treinamento. 

Esta rede é munida de estatísticas para auxiliar em análises de desempenho de atletas em geral e não 
somente ciclistas. Suas rotas têm como finalidade 
demarcar o trajeto e progressos físicos em rotinas de treinos e, assim como as outras funções desta rede, 
não abrangem atividades turísticas. Possui o 
recurso de enviar fotos de momentos durante sessões de atividades físicas realizadas pelo usuário e duas opções de idiomas, inglês e alemão. Permite 
utilizar aplicativo \textit{mobile} para acompanhamento e GPS para gerar as estatísticas com base na atividade realizada. 

Como recursos de rede social, o Runtastic permite relacionamento de amizade entre usuários, \textit{posts} no mural de outros usuários, 
acompanhamento em tempo real por mapa da atividade realizada por outros usuários, enviar \textit{cheers} para o usuário que está 
realizando a atividade, além de permitir curtir e comentar a atividade realizada. 

\subsection{Strava}
A rede Strava \cite{strava} possui diversas funções voltadas à prática de atividades físicas diárias, focadas em caminhadas/corridas e ciclismo. 
Possibilita que seus usuários ao redor do mundo, registrem seus esforços físicos diariamente, pois possui um registro de atividades físicas visando 
comparar o seu progresso. 

Esta rede oferece a seus usuários desafios para estimulá-los a aumentar sua performance, possui o recurso de traçar rotas com o objetivo de planejar 
a atividade esportiva e treinamento e, é possível cadastrar metas personalizadas relacionadas ao desempenho, permitindo que seus usuários 
acompanhem o progresso de suas atividades. Também possui quadros informativos sobre distâncias percorridas em relação a espaços de tempo. 

Como recursos de rede social, o Strava permite ao usuário seguir outros usuários, ingressar em grupos de usuários 
denominados ``clubes'' que podem ser públicos ou privados, possui um \textit{feed} de atividades onde 
são exibidas as atividades realizadas por todos os 
usuários da rede. Neste \textit{feed} são exibidas fotos compartilhadas durante a realização destas atividades e a rota percorrida com informações 
de distância total, horário inicial e final da atividade. Esta rede possui um aplicativo \textit{mobile} que registra, utilizando o GPS, a 
atividade do usuário e, permite que o mesmo compartilhe fotos tiradas durante o percurso. O Strava não oferece nenhum recurso voltado a atividades 
turísticas.

No Quadro 1 podemos ver a comparação entre as funções do Ciclotour e das redes analisadas.

   \begin{center}
          \centering
       	\textbf{Quadro 1 - Comparação entre funcionalidades das Redes e o Ciclotour}\\
        \label{quadro1}
	\begin{tabular}{|c|c|c|c|} \hline
	\multicolumn{4}{|c|}{\textbf{Comparação entre funcionalidades das Redes e o Ciclotour }} 	  \\
		\hline \textbf{ Funções } & Strava & Runtastic & Ciclotour \\  
		\hline \textbf{ Criar rotas personalizadas } & SIM & SIM & SIM \\ 
		\hline \textbf{ Criar pontos informativos em rotas } & NÃO & NÃO & SIM  \\
		\hline \textbf{ Indicar tipo do terreno da rota } & SIM & SIM & SIM \\ 
		\hline \textbf{	Adicionar Fotos à Rota } & NAO & NAO & SIM \\ 
		\hline \textbf{ Comentários em rotas }	& NÃO & SIM & SIM \\ 
		\hline \textbf{	Comentários em pontos } & NAO & NAO & SIM \\ 
		\hline \textbf{	Marcar rotas como realizadas ou pendentes } & NAO & NAO & SIM \\ 
		\hline \textbf{	Permitir relação de amizade entre usuários } & NÃO & SIM & SIM \\ 
		\hline \textbf{	Buscar rotas com base em origem e destino } & NÃO & NÃO & SIM \\ 
		\hline
	\end{tabular}
	\vspace{0.1cm} 
	{\footnotesize\\ \textbf{Fonte: Elaborado pelo autor}}
   \end{center}

\section{\esp Metodologia}
O ciclo de desenvolvimento da aplicação Ciclotour foi dividido nas seguintes fases: 1) coleta inicial de requisitos; 2) teste de aceitação via 
\textit{mockup} por parte do fornecedor de requisitos; 3) implementação das \textit{features} aceitas; 4) teste final de aceitação. Para isso, 
o desenvolvimento do Ciclotour contou com uma abordagem \textit{top-down}. Nesta abordagem, o \textit{front-end} é construído antes do 
\textit{back-end}. Foi adotada esta abordagem de desenvolvimento pois, deste modo, é possível aprovar o funcionamento do sistema a partir de um 
\textit{mockup} demonstrando como seria a usabilidade de cada tela do sistema, sem a necessidade de implementar toda a aplicação. 
Assim, o ``custo'' de mudanças 
é reduzido, pois além do que foi contruído passar por uma pré-aprovação do fornecedor de requisitos, o maior custo de mudanças acontece em 
alterações de modelos de banco de dados, que só será implementado após aprovação, evitando que seja necessária este tipo de mudança. Deste modo, é 
possível evitar o retrabalho. Nas próximas seções, serão descritas as fases do ciclo de desenvolvimento da aplicação Ciclotour.

\subsection{Coleta inicial de requisitos}
Na fase de coleta inicial de requisitos, foi realizada uma reunião com cicloturistas que são os principais fornecedores de requisitos para a 
aplicação. Nesta fase, foram expostos os principais problemas enfrentados pelos fornecedores durante o planejamento de sua atividade, as 
ferramentas que eram utilizadas para auxiliar na solução destes problemas, além de informações que os fornecedores de requisitos tinham maior
dificuldade de encontrar. Após esta reunião, foram realizadas as análises destes requisitos e, criada 
uma proposta inicial de solução apresentada aos envolvidos. 
Após a aprovação desta proposta, iniciou-se a contrução do \textit{mockup} que foi utilizado na próxima fase.

\subsection{Teste de aceitação do \textit{Mockup}}
Após finalizar a construção do \textit{mockup}, onde é possível interagir de forma prototipada com o sistema, percebendo o funcionamento das 
\textit{features} envolvidas, usabilidade e o \textit{workflow} do sistema, é realizado um teste de aceitação por parte dos 
cicloturistas envolvidos no projeto para antecipar o \textit{feedback} em relação às mudanças necessárias, com o objetivo de adequar a aplicação às 
espectaticas do usuário. Esta fase foi executada em várias iterações, até que o \textit{mockup} fosse totalmente aprovado pelos fornecedores de 
requisitos e, somente então, dá-se início à construção do \textit{back-end}.

\subsection{Implementação das \textit{features} aceitas}
Com o protótipo aprovado, iniciou-se a fase de implementação das \textit{features} do Ciclotour. Nesta fase, foi utilizada a técnica conhecida 
como \textit{Test Driven Development} (TDD). Esta técnica tem como princípio o desenvolvimento guiado por testes, visando garantir que, o que foi 
desenvolvido está correto, evitando, assim, o retrabalho. Utilizando esta técnica, é possível garantir a conformidade com os requisitos 
e a prevenção de \textit{bugs} que possam surgir. 

\subsection{Teste final de aceitação}
Cada \textit{feature} implementada gerou um \textit{deploy} da aplicação para o servidor, onde foi possível que os fornecedores de requisitos testassem
e retornassem um \textit{feedback} o quanto antes, evitando maiores impactos por conta de mudanças necessárias. Essa fase junto à fase de 
implementação ocorreu de forma iterativa, até que o sistema estivesse totalmente construído.

Este ciclo de desenvolvimento foi escolhido para desenvolver a aplicação Ciclotour pois, deste modo, foi possível minimizar o retrabalho ao 
máximo, o que no caso deste trabalho era de extrema importância devido ao curto período de tempo disponível para realizá-lo. 

\subsection{Ferramentas}

\subsubsection{Servidor}
Para desenvolver o \textit{server-side} da aplicação Ciclotour, foram utilizados o \textit{framework} Python de desenvolvimento \textit{Web} Django com a 
extensão Django REST Framework para construção de \textit{\textit{Web} Services}. O Django utiliza um estilo arquitetural baseado no 
\textit{Model-View-Controller} (MVC), o \textit{Model-View-Template} (MVT). Na aplicação Ciclotour, o Django é responsável por processar a 
solicitação inicial, onde a página com todos os \textit{scripts} necessários para a aplicação é devolvida como resposta para o solicitante,
manter uma interface de administração do sistema e, manter os modelos do sistema por meio de seu \textit{Object-relational mapping} (ORM). 
As requisições posteriores enviadas ao servidor são requisições de dados que serão processadas e respondidas pelo Django REST \textit{framework} 
em formato \textit{JavaScript Object Notation} (JSON).

\subsubsection{Cliente}
Para desenvolver o \textit{client-side} da aplicação Ciclotour foram utilizados o \textit{framework} JavaScript AngularJS e o \textit{framework} 
CSS Bootsrap, além da API JavaScript do \textit{GoogleMaps}. 
O AngularJS tem como propósito permitir a construção de elementos \textit{client-side} de forma 
declarativa, por meio de suas diretivas. 
Este \textit{framework} também adota o estilo arquitetural MVC, além de prover uma abstração para as chamadas
Ajax e atualização em tempo real do \textit{Document Object Model} (DOM) através do \textit{Two-way Data Binding}, o que promove uma melhor 
organização do projeto que, por se tratar de uma SPA, possui alto grau de complexidade em seus scripts. Devido a esta facilidade oferecida pelo 
\textit{framework}, o AngularJS foi utilizado para desenvolver de todos os scripts \textit{client-side} da aplicação Ciclotour. 
O Bootstrap foi utilizado 
com o intuito de oferecer uma interface rica e com \textit{design} responsivo, promovendo melhor usabilidade e experiência por parte do usuário 
final do sistema. Para prover recursos geográficos que possibilitem a interação do usuário de forma intuitiva, a aplicação Ciclotour conta com a API 
Javascript de mapas do Google, o \textit{GoogleMaps}, que oferece recursos gráficos para 
renderização de mapas interativos, além de recursos para obter menor 
distância entre pontos e diversos recursos geográficos.

\section{\esp Aplicação Ciclotour}
O Ciclotour foi desenvolvido com o principal objetivo de atender às demandas específicas percebidas no meio do cicloturismo, uma atividade que exige
um extenso planejamento por parte do praticante e que carece de meios de informações focados em suas atividades. Os requisitos para esta aplicação
são: a aplicação deve ser construída em formato rede social 100\% focada em cicloturismo, com o propósito de centralizar informações sobre 
cicloturismo. Nesta aplicação, deve ser possível cadastrar rotas personalizadas, de modo que os próprios usuários alimentem a base de dados com 
o trajeto, os lugares para dormir, para comer, pontos turísticos interessantes, lugares que não valem a pena passar e etc. Deve permitir que 
os usuários expressem as suas opiniões, façam observações e enviem fotos sobre as rotas e lugares.

Para tal, o Ciclotour foi desenvolvido adotando uma arquitetura centrada em dados, de modo a se tornar um grande repositório de informações 
específicas ao meio do cicloturismo alimentado pelos próprios usuários de forma colaborativa. Assim, será possível centralizar informações que 
estão disponíveis de forma esparsa na \textit{Web} em um único local e auxiliar os cicloturistas em seu planejamento sem que 
o mesmo precise extender sua pesquisa às infinitas fontes espalhadas pela \textit{internet}. Para alcançar este objetivo, o Ciclotour 
foi desenvolvido no formato de rede social, para 
estimular o compartilhamento de informações por parte de seus usuários, através de interações por comentários, cadastro de rotas personalizadas, 
cadastro de pontos em rotas e envio de fotos para rotas. Nas seções seguintes serão detalhadas todas as funcionalidades da aplicação.

\subsection{Arquitetura do sistema}
Para o desenvolvimento do Ciclotour, foi utilizado o estilo arquitetural RESTful, sendo este sistema contruído como uma SPA. Deste modo, há
uma separação completa das responsabilidades entre cliente e servidor, onde a responsabilidade de composição das páginas é delegada ao 
\textit{browser} executado \textit{client-side} e o servidor passa a atuar apenas como um repositório dos dados pertinentes à aplicação. A Figura 
\ref{fig:estruturaCiclotour} mostra a estrutura simplificada entre cliente e servidor da aplicação Ciclotour, e as tecnologias utilizadas para
a construção do sistema.

\begin{figure}[!ht]
	\centering	
	\caption[\hspace{0.1cm} Estrutura simplificada entre cliente e servidor da aplicação Ciclotour.]
	{Estrutura simplificada entre cliente e servidor da aplicação Ciclotour}
	  \vspace{-0.4cm}
	\includegraphics[width=1\textwidth]{figuras/estruturaCiclotour.png}
	 \vspace{-0.3cm}
	\\\textbf{\footnotesize Fonte: Elaborada pelo autor}
	\label{fig:estruturaCiclotour}
\end{figure}

\subsection{Cadastro de rotas}
A aplicação Ciclotour conta com uma funcionalidade de cadastro de rotas personalizadas, possibilitando ao usuário traçar rotas de forma interativa 
por meio de simples cliques em um mapa. Para tal, basta que o usuário informe uma origem e após preencher este campo, será exibido um mapa na 
tela da aplicação centralizado na origem informada pelo usuário. Para continuar o cadastro da rota, basta que o usuário clique no mapa marcando os 
pontos para traçar a rota conforme demonstrado na Figura \ref{fig:cadastroRotas}. 

\begin{figure}[!ht]
	\centering	
	\caption[\hspace{0.1cm} Cadastro de Rotas da aplicação Ciclotour.]
	{Formulário de cadastro de rotas da aplicação Ciclotour}
	  \vspace{-0.4cm}
	\includegraphics[width=0.8\textwidth]{figuras/cadastroRotas.png}
	 \vspace{0cm}
	\\\textbf{\footnotesize Fonte: Elaborada pelo autor}
	\label{fig:cadastroRotas}
\end{figure}

O Ciclotour conta com duas formas de traçar a rota entre estes pontos, sendo a forma automática, que permite ao usuário traçar a melhor rota que 
interliga estes pontos, fornecida pela API do \textit{GoogleMaps}, ou a forma manual, onde a rota entre os pontos será uma linha reta. Além destas 
informações, o usuário deve fornecer um título para esta rota, uma descrição, e a informação do tipo de terreno desta rota. Os tipos de 
terreno disponíveis inicialmente no sistema 
são: montanhoso, estrada, estrada de terra, trilha e praia. Novos tipos de terreno podem ser cadastrados na interface de administração do sistema 
conforme for necessário.

Feito isto, o usuário tem a possibilidade de restringir o acesso desta rota, visando garantir sua privacidade. As opções disponíveis para privacidade 
de rotas são: privada, pública e restrita aos amigos. Na opção privada, somente o usuário terá acesso à esta rota, 
enquanto, na opção pública, todos os 
usuários do Ciclotour podem interagir com a rota. Além destas opções, o Ciclotour ainda permite ao usuário configurar a rota como restrita a amigos, 
de modo que somente os amigos do usuário que criou a rota poderão interagir com a mesma.

\subsection{Cadastro de pontos em rotas}
Uma informação muito importante para os cicloturistas é o que poderão encontrar durante o seu percurso nas rotas. É de extrema importância para um 
cicloturista saber por exemplo, onde ele pode encontrar pontos de paradas como restaurantes, pousadas, hotéis, pois, muitas vezes, 
as rotas percorridas
por estas pessoas é extensa e podem, inclusive, durar dias. Para isto, a aplicação Ciclotour possui um cadastro de pontos informativos em rotas, onde
os usuários podem cadastrar os pontos em determinada rota, bastando clicar sobre ele no mapa da rota e informar um título, o tipo do ponto e uma 
descrição. O sistema irá gerar, automaticamente, o endereço do ponto, conforme visto na Figura \ref{fig:cadastroPonto}. 

\begin{figure}[!ht]
	\centering	
	\caption[\hspace{0.1cm} Cadastro de Pontos em Rotas da aplicação Ciclotour.]
	{Formulário de cadastro de ponto em uma rota da aplicação Ciclotour}
	  \vspace{-0.4cm}
	\includegraphics[width=0.8\textwidth]{figuras/cadastroPonto.png}
	 \vspace{0cm}
	\\\textbf{\footnotesize Fonte: Elaborada pelo autor}
	\label{fig:cadastroPonto}
\end{figure}

O sistema inicialmente conta com 6 tipos de pontos, sendo eles: natureza exuberante, pontos turísticos, estadia, restaurante, alerta e proibição. 
Cada tipo possui um ícone que é exibido no mapa, facilitando ao usuário a identificação do tipo do ponto. 
Na interface de administração do sistema, é possível 
cadastrar novos tipos de pontos, de acordo com a necessidade. Estes tipos de pontos iniciais do sistema foram levantados com o fornecedor de 
requisitos como sendo os principais pontos de interesse para cicloturistas em uma rota.

\subsection{Comentários em pontos e rotas}
Fóruns na \textit{Web} geram uma grande quantidade de informações a partir da interação entre os usuários. 
Pensando nisto, a aplicação Ciclotour conta com uma 
seção de comentários em rotas e em pontos, possibilitando a interação entre os usuários e permitindo ao usuário expor suas observações e pontos de 
vista, gerando informações relativas a estes pontos e rotas que podem ser aproveitadas por outros usuários, além de ser um local onde os 
usuários podem tirar as suas dúvidas com outros usuários que já realizaram tal rota ou passaram em tal ponto.

\subsection{Buscar rotas}
Ao entrar na página de buscas de rotas, são exibidas as rotas cadastradas ordenadas por sua data de cadastro, exibindo as mais recentes.
O Ciclotour possui a opção busca por rotas a partir de origem e destino, o qual o usuário pode informar de onde deseja partir e até onde deseja 
chegar, o sistema irá retornar as rotas já cadastradas que atendam a esta condição. Para esta funcionalidade, foi utilizado um raio de 5 km entre 
as coordenadas do ponto de origem e destino informados pelo usuário. Ao informar origem e destino, o sistema automaticamente converte o texto em 
coordenadas utilizando a API do \textit{GoogleMaps}, e realiza um cálculo de raio de 5 km e busca as rotas cadastradas
que possuem seu primeiro e último pontos
dentro do raio de 5km da origem e destino informados pelo usuário respectivamente. Este cálculo foi feito utilizando a equação matemática reduzida da 
circunferência e foi necessário descobrir a correspondência de latitude/longitude em quilômetros para implementá-lo. Esta funcionalidade pode ser 
vista na Figura \ref{fig:buscarRotas}.

\begin{figure}[!ht]
	\centering	
	\caption[\hspace{0.1cm} Buscar Rotas.]
	{Página das rotas que atendam a busca realizada}
	  \vspace{-0.4cm}
	\includegraphics[width=0.8\textwidth]{figuras/buscarRotas.png}
	 \vspace{0cm}
	\\\textbf{\footnotesize Fonte: Elaborada pelo autor}
	\label{fig:buscarRotas}
\end{figure}

\subsection{Marcar rotas como realizadas ou pendentes}
A aplicação Ciclotour permite ao usuário marcar as rotas como realizadas ou pendentes conforme mostra 
a Figura \ref{fig:marcarRota}, possibilitando organizar
suas próximas atividades por meio das rotas pendentes, os quais o usuário deseja realizar. Além de permitir marcar as
rotas como pendentes, o usuário pode
marcá-las como realizadas e a quantidade de rotas pendentes e realizadas pelo usuário são exibidas para os outros usuários do Ciclotour.

\begin{figure}[!ht]
	\centering	
	\caption[\hspace{0.1cm} Marcar rotas como realizadas ou pendentes.]
	{Página de visualização da rota onde é possível o usuário marcá-la como realizada ou pendente}
	  \vspace{-0.4cm}
	\includegraphics[width=0.8\textwidth]{figuras/marcarRota.png}
	 \vspace{0cm}
	\\\textbf{\footnotesize Fonte: Elaborada pelo autor}
	\label{fig:marcarRota}
\end{figure}

\subsection{Relação de amizade}

O Ciclotour permite relações de amizades entre usuários mediante solicitação. Usuários podem buscar por outros usuários pelo seu nome, 
sobrenome e 
endereço de e-mail e, enviar solicitação de amizade a este usuário que pode ou não aceitá-la, como demonstra a Figura \ref{fig:buscarUsuario} e 
Figura \ref{fig:solicitacaoAmizade}. A relação de amizade entre os usuários permitem que os mesmos vejam em seus \textit{feeds} as atividades de seus 
amigos. O Ciclotour também possui um contador de amizades que será exibido para outros usuários.

\begin{figure}[!ht]
	\centering	
	\caption[\hspace{0.1cm} Buscar usuários.]
	{Página de busca de usuários e envio de solicitação de amizade}
	  \vspace{-0.4cm}
	\includegraphics[width=0.8\textwidth]{figuras/buscarUsuario.png}
	 \vspace{0cm}
	\\\textbf{\footnotesize Fonte: Elaborada pelo autor}
	\label{fig:buscarUsuario}
\end{figure}

\begin{figure}[!ht]
	\centering	
	\caption[\hspace{0.1cm} Solicitações de amizade.]
	{Página de onde são exibidas as solicitações de amizade}
	  \vspace{-0.4cm}
	\includegraphics[width=0.8\textwidth]{figuras/solicitacaoAmizade.png}
	 \vspace{0cm}
	\\\textbf{\footnotesize Fonte: Elaborada pelo autor}
	\label{fig:solicitacaoAmizade}
\end{figure}

\subsection{Adicionar fotos em rotas}
Em se tratando de uma aplicação voltada ao turismo, é importante que os usuários possam compartilhar 
fotos de suas atividades com outros usuários. Fotos 
podem despertar interesse de outros cicloturistas pela rota, além de ser uma fonte de informação. É muito comum usuários postarem fotos em pontos 
turísticos em outras redes sociais. Sendo o Ciclotour uma rede social focada em turismo, este recurso é indispensável. Para acessá-lo, basta buscar
pela rota que deseja adicionar a foto, clicar no botão "Adicionar Foto", como pode ser visto na Figura \ref{fig:marcarRota}, então abrirá um 
\textit{popup} onde o usuário poderá fazer o \textit{upload} da imagem desejada, além de informar uma descrição para a mesma conforme Figura 
\ref{fig:adicionarFoto}. Após enviar a foto, a mesma será adicionada nas imagens da rota, de modo que os usuários que têm permissão para acessar 
esta rota poderão visualizá-la, conforme a Figura \ref{fig:fotosRota}. Ao clicar sobre a imagem, a mesma é ampliada e é possível ver a descrição 
informada pelo usuário que a enviou.

\begin{figure}[!ht]
	\centering	
	\caption[\hspace{0.1cm} Adicionar Foto.]
	{Popup para adicionar foto à rota}
	  \vspace{-0.4cm}
	\includegraphics[width=0.8\textwidth]{figuras/adicionarFoto.png}
	 \vspace{0cm}
	\\\textbf{\footnotesize Fonte: Elaborada pelo autor}
	\label{fig:adicionarFoto}
\end{figure}

\begin{figure}[!ht]
	\centering	
	\caption[\hspace{0.1cm} Fotos da Rota.]
	{Página de onde são exibidas as fotos de uma rota}
	  \vspace{-0.4cm}
	\includegraphics[width=0.8\textwidth]{figuras/fotosRota.png}
	 \vspace{0cm}
	\\\textbf{\footnotesize Fonte: Elaborada pelo autor}
	\label{fig:fotosRota}
\end{figure}

\section{\esp Conclusão}
Este artigo apresentou a aplicação Ciclotour, uma rede social colaborativa para cicloturistas, desenvolvida utilizando os \textit{frameworks} Django 
e AngularJS, cujo objetivo principal é a construção da primeira aplicação 100\% focada em cicloturismo e 
capaz de proporcionar aos seus usuários um local onde 
possam interagir com outros praticantes desta atividade, compartilhar e encontrar informações que auxiliem o planejamento de suas atividades, 
oferecendo recursos para atender as necessidades específicas deste meio percebidas por este trabalho e que ainda não foram exploradas por 
outras aplicações.

Para atingir estes objetivos, foi necessário mesclar recursos das aplicações voltadas a prática esportiva, especificamente o ciclismo utilizadas pelos
cicloturistas, com recursos interessantes para prática do turismo. Esta aplicação foi desenvolvida no formato de \textit{Single Page Application}, 
seguindo as novas tendências das redes sociais modernas como o Facebook e o Instagram e que melhora a experiência do usuário utilizando a aplicação.

Dado o tempo para o desenvolvimento da aplicação, o foco foi desenvolver uma primeira versão que implementasse as \textit{features} mais essenciais
com uma interface gráfica básica, de modo que os usuários pudessem utilizá-la para resolver o problema proposto, porém preparada para receber 
extensões que agregariam novas possibilidades ao sistema e melhoria da usabilidade da aplicação. Estas extensões e melhorias estão previstas 
para trabalhos futuros, dado que o necessário para que usuários utilizem o sistema já está implementado. 

Como trabalhos futuros na aplicação Ciclotour, pretende-se uma melhoria de \textit{design} para atender à requisitos de \textit{User Experience} (UX), 
que são fundamentais para sistemas no formato rede social, por exemplo, melhoria na visualização dos mapas por meio de um replanejamento das 
interfaces da aplicação e melhoria no \textit{feedback} para o usuário, com mensagens mais detalhadas sobre falhas, alertas e sucesso. Além desta 
melhoria, pretendemos também passar a utilizar a extensão para trabalhar com dados geográficos no banco de dados da aplicação, o PostGIS, para 
melhorar o gerenciamento deste tipo de dados. Com esta mudança será possível oferecer ainda mais recursos geográficos na aplicação e melhorar os 
que já são oferecidos, visto que esta extensão permite trabalhar com dados geográficos com maior precisão além de oferecer recursos de relações 
topológicas que SGBDs convencionais não oferecem. Pretende-se também construir um aplicativo \textit{mobile} para ampliar as possibilidades da rede, 
permitindo a utilização de recursos desta plataforma, como o GPS, integrados às funcionalidades da aplicação. Está previsto também a implementação de 
uma \textit{feature} que inclua informações sobre o clima e marés nas rotas, pois estas informações são fundamentais para alguns tipos de trajetos.



