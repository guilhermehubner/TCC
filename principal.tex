\documentclass[a4paper,12pt,Times]{article}
\usepackage{abakos}  %pacote com padrão da Abakos baseado no padrão da PUC

%%%%%%%%%%%%%%%%%%%%%%%%%%%
%Capa da revista
%%%%%%%%%%%%%%%%%%%%%%%%%%

%\setcounter{page}{80} %iniciar contador de pagina de valor especificado
\newcommand{\monog}{Ciclotour - Uma rede social colaborativa para cicloturistas}
\newcommand{\monogES}{Ciclotour - A colaborative social network for ciclotourists}
\newcommand{\tipo}{Artigo }  % Especificar a seção tipo do trabalho: Artigo, Resumo, Tese, Dociê etc
\newcommand{\origem}{Brasil }
\newcommand{\editorial}{Belo Horizonte, p. 01-03, abr. 2016}  % p. xx-xx – páginas inicial-final do artigo
\newcommand{\lcc}{\scriptsize{Licença Creative Commons Attribution-NonCommercial-NoDerivs 3.0 Unported}}

%%%%%%%%%%%%%%%%%INFORMAÇÕES SOBRE AUTOR PRINCIPAL %%%%%%%%%%%%%%%%%%%%%%%%%%%%%%%
\newcommand{\AutorA}{Guilherme Hübner Franco}
\newcommand{\funcaoA}{}
\newcommand{\emailA}{guilherme\underline{\space}hubner@msn.com}
\newcommand{\cursA}{Curso de Graduação em Sistemas de Informação da PUC Minas}
% 
% Definir macros para o nome da Instituição, da Faculdade, etc.
\newcommand{\univ}{Pontifícia Universidade Católica de Minas Gerais}

\newcommand{\keyword}[1]{\textsf{#1}}

\begin{document}
% %%%%%%%%%%%%%%%%%%%%%%%%%%%%%%%%%%
% %% Pagina de titulo
% %%%%%%%%%%%%%%%%%%%%%%%%%%%%%%%%%%

\begin{center}
\includegraphics[scale=0.2]{figuras/brasao.jpg} \\
PONTIFÍCIA UNIVERSIDADE CATÓLICA DE MINAS GERAIS \\
Instituto de Ciências Exatas e de Informática

% \vspace{1.0cm}

\end{center}

 \vspace{0cm} {
 \singlespacing \Large{\monog \symbolfootnote[1]{Artigo apresentado ao Instituto de Ciências Exatas e Informática da Pontifícia Universidade Católica de Minas Gerais como pré-requisito para obtenção do título de Bacharel em Sistemas de Informação.} \\ }
  \normalsize{\monogES}
 }

\vspace{1.0cm}

\begin{flushright}
\singlespacing 
\normalsize{\AutorA \footnote{\funcaoA \cursA, \origem -- \emailA . }} \\
% \normalsize{\AutorB \footnote{\funcaoB, E-mail:\emailB \\ \cursB, \origem. }} \\
% \normalsize{\AutorC \footnote{\funcaoC, E-mail:\emailC \\ \cursC, \origem. }} \\
% \normalsize{\AutorD \footnote{\funcaoD \\ Pais de origem: \origemD. E-mail: \emailD}} \\
%deixar com o valor `0` e usar o '*' no inicio da frase
% \symbolfootnote[0]{Artigo recebido em 10 de julho de 1983 e aprovado em 29 de maio 2012}
\end{flushright}
\thispagestyle{empty}

\vspace{1.0cm}

\begin{abstract}
\noindent
Este trabalho tem como objetivo o desenvolvimento de uma rede social com o foco em 
cicloturismo, onde os usuários podem à partir da interação com o sistema, ter acesso à informações 
relevantes em relação às rotas que pretendem percorrer, auxiliando no planejamento da atividade 
turística. Esta rede social seguirá o modelo centrado em dados onde os usuários serão responsáveis 
por alimentar a base de dados do sistema de forma colaborativa, assim será possível centralizar 
informações sobre as rotas. 
\\\textbf{\keyword{Palavras-chave: }} Rede Social, Cicloturismo.
\end{abstract}

%%%%%%%%%%%%%%%%%%%%%%%%%%%%%%%%%%%%%%%%%%%%%%%%%%%%%%%%%
%\newpage    %%%% CASO QUEIRA QUE O RESUMO FIQUE EM UMA PAGINA E O ABSTRACT EM OUTRA
%\selectlanguage{english}
%\begin{abstract}
%\noindent
%The present work has as a goal the development of a social social network focus on bycicle touring, in which the users can have
%access to relevant information about the routes they intend to take through interactions with the system, assisting on their touristic
%activity planning. This social network will follow the data-center model in which users are responsible to feed the database of the system 
%collaboratively, so this way it's possible to centralize information about the routes.
%\\\textbf{\keyword{Keywords: }} Social Network, Bycicle Touring.
%\end{abstract}

\selectlanguage{brazilian}
 \onehalfspace  % espaçamento 1.5 entre linhas
 \setlength{\parindent}{1.25cm}

%%%%%%%%%%%%%%%%%%%%%%%%%%%%%%%%%%%%%%%%%%%%%%%%%
%% INICIO DO TEXTO
%%%%%%%%%%%%%%%%%%%%%%%%%%%%%%%%%%%%%%%%%%%%%%%%%

\section{\esp Introdução}
Com a mudança de mentalidade da Web 1.0 para o novo paradgma de aplicações web conhecido como Web 2.0 proposta por \cite{web20Proposta}, seus 
usuários deixaram de ter um papel passivo como consumidores de conteúdo para assumir um papel ativo de produtores de conteúdo. Esta mudança 
trouxe consigo ferramentas interativas, que tem como foco principal a geração de conteúdo por parte dos próprios usuários, tais como fóruns 
de discursões e recentemente, as redes sociais como o Twitter, Facebook e o Instagram.

Esta participação ativa dos usuários impactou singnificativamente a quantidade de informação disponíveis na Web. Segundo \cite{artigo01}, 
o tráfego de dados nesta rede dobra a cada ano, o que gera uma quantidade massiva de informação, tornando a Web muito influente em diversos setores 
de grande importância na sociedade, passando a fazer parte do cotidiano das pessoas que estão cada vez mais conectadas a mecanismos que as 
conectam aos conteúdos de seu interesse. Podemos perceber esta influência principalmente através das redes sociais, que segundo 
\cite{redesSociais01} está mudando profundamente as formas de organização, identidade, conversação e mobilização social.

Como descreve \cite{deitelAjax}, com todo este conteúdo disponível, os mecanismos de busca de informação 
acabam recebendo destaque nesta ``nova Web'' e sobresaem aqueles que auxiliam os usuários  a localizar e filtrar 
as informações desejadas. Desta forma, há um grande desafio em relação à formulação de ferramentas que guie o usuário até os conteúdos de seu 
interesse, economizando assim sua atenção. Estas ferramentas com o passar do tempo se fazem mais presentes no dia-a-dia das pessoas, 
que por sua vez demandam por mecanismos cada vez mais robustos e personalizados de acordo com suas necessidades, como consequência existe 
uma demanda constante de novas aplicações mais específicas para atender determinada comunidade.

O cicloturismo conforme descrito por \cite{cicloturismo01} é uma modalidade do ecoturismo que está ganhando cada vez mais adeptos no Brasil,
por ser uma atividade de baixo impacto ambiental, já que é realizado com bicicletas. Neste meio, há uma demanda ainda não explorada por 
uma aplicação focada nesta prática, pois de acordo com \cite{cicloturismo02}, as grandes viagens realizadas sobre a bicicleta requerem um melhor
preparo e conhecimento, por parte do ciclista, e ressalta a importância de se conhecer a extensão da viagem e tempo total disponível, além da 
região, relevo e clima escolhidos como trajeto. Devido a complexidade do planejamento requerido, percebe-se neste meio a necessidade de 
uma ferramenta que centralize informações importantes para tal. Praticantes desta atividade por não possuírem uma ferramenta específica, 
recorrem a outras com propósito similares para auxiliá-los no planejamento de seus trajetos, que não oferecem informações importantes para este 
planejamento.

Este trabalho tem como objetivo desenvolver uma ferramenta no formato rede social Web voltada a atender demandas observadas no meio do cicloturismo,
assim usuários desta ferramenta terão acesso de forma centralizada a informações sobre rotas que pretendem trilhar, fornecidas por outros usuários
de forma colaborativa, com o propósito de criar um grande repositório de informações relevantes para cicloturistas auxiliando no planejamento da 
atividade turística.

%%%%%%%%%%%%%%%%%%%%%%%%%%%%%%%%%%%
%% FIM DO TEXTO
%%%%%%%%%%%%%%%%%%%%%%%%%%%%%%%%%%%

% \selectlanguage{brazil}
%%%%%%%%%%%%%%%%%%%%%%%%%%%%%%%%%%%
%% Inicio bibliografia
%%%%%%%%%%%%%%%%%%%%%%%%%%%%%%%%%%%

\newpage
\singlespace{

\bibliographystyle{abnt-alf}
\bibliography{bibliografia}

}

\end{document}


