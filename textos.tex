\section{\esp Introdução}
Com a evolução da Web para sua versão 2.0, seus usuários deixaram de ter um papel passivo como consumidores de conteúdo para assumir
um papel ativo de produtores de conteúdo. Esta evolução trouxe consigo ferramentas interativas, que tem como foco principal a geração de 
conteúdo por parte dos próprios usuários, tais como fóruns de discursões e recentemente, as redes sociais como o Twitter, Facebook 
e o Instagram.

Esta participação ativa dos usuários impactou singnificativamente a quantidade de informação disponíveis na Web. Segundo \cite{artigo01}, 
o tráfego de dados nesta rede dobra a cada ano, o que gera uma quantidade massiva de informação, tornando a Web muito influente em diversos setores 
de grande importância na sociedade, passando a fazer parte do cotidiano das pessoas que estão cada vez mais conectadas a mecanismos que as 
conectam aos conteúdos de seu interesse. Podemos perceber esta influência principalmente através das redes sociais, que segundo 
\cite{redesSociais01} está mudando profundamente as formas de organização, identidade, conversação e mobilização social.

Como descreve \cite{deitelAjax}, com todo este conteúdo disponível, os mecanismos de busca de informação 
acabam recebendo destaque nesta ``nova Web'' e sobresaem aqueles que auxiliam os usuários  a localizar e filtrar 
as informações desejadas. Desta forma, há um grande desafio em relação à formulação de ferramentas que guie o usuário até os conteúdos de seu 
interesse, economizando assim sua atenção. Estas ferramentas com o passar do tempo se fazem mais presentes no dia-a-dia das pessoas, 
que por sua vez demandam por mecanismos cada vez mais robustos e personalizados de acordo com suas necessidades, como consequência existe 
uma demanda constante de novas aplicações mais específicas para atender determinada comunidade.

No meio do cicloturismo, há uma demanda ainda não explorada por uma aplicação focada nesta prática, que centralize informações 
importantes para tal. Praticantes desta atividade por não possuírem uma ferramenta específica, recorrem a outras com propósito similares 
para auxiliá-los no planejamento de seus trajetos.

Este trabalho tem como objetivo desenvolver uma ferramenta no formato rede social Web voltada a atender demandas observadas no meio do cicloturismo,
assim usuários desta ferramenta terão acesso de forma centralizada a informações sobre rotas que pretendem trilhar, fornecidas por outros usuários
de forma colaborativa, com o propósito de criar um grande repositório de informações relevantes para cicloturistas auxiliando no planejamento da 
atividade turística.